\chapter*{ВВЕДЕНИЕ}
\addcontentsline{toc}{chapter}{ВВЕДЕНИЕ}

Базы данных в современном цифровом мире – неотъемлемая часть любой организации, желающая обеспечить высокую производительность своего предприятия, путем организации быстрого, удобного и надежного доступа к данным, необходимых для протекания бизнес-процессов. 

Системы управления базами данных (СУБД) – совокупность программных и лингвистических средств общего или специального назначения, обеспечивающих управление созданием и использованием баз данных. От того, как спроектирована база данных, зависит производительность предприятия, а также безопасность его данных.

Целью выполнения курсовой работы является применение на практике знаний, полученных в процессе изучения дисциплины «Безопасность систем баз данных», и получение практических навыков создания защищенных автоматизированных информационных систем (АИС), основанных на базах данных.

Результатом выполнения курсовой работы является база данных, созданная с помощью современной СУБД и являющаяся основой автоматизированной системы, а также разработанный комплекс средств обеспечения безопасности информационных ресурсов.
Знакомство с теорией построения баз данных было произведено с помощью источника [1].
